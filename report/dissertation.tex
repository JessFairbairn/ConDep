\documentclass{article}
\usepackage{booktabs}
\usepackage{graphicx}
\usepackage{subfiles}
\usepackage{natbib}
\usepackage{graphicx}
\usepackage{hyperref}
\usepackage{mathtools}
\usepackage{algpseudocode}
\usepackage{booktabs}
\usepackage{rotating}
\usepackage{pifont}
\usepackage{svg}
\usepackage{float}

\providecommand{\cdpred}{./cd-pred-arg} % relative path to master.tex
\providecommand{\repsfolder}{./reps}

\author{Jess Fairbairn}
\title{Dissertation}

\begin{document}
    \maketitle
    \tableofcontents
    \section{Introduction}
    This dissertation will begin by looking at conceptual dependency theory, a schema for expressing the semantics of natural language originally put forward by Roger Schank~\cite{SCHANK1972552}. This will then be compared to a number of other forms of information representation, with the aim of identifying what aspects of meaning each is able to capture.

    The code developed for this project will then be discussed, explaining how simulated events were ran in terms of conceptual dependencies. 

    \section{Background}
    \subfile{reps/paper.tex}
    \subfile{conceptual_dependancies}

    \section{Implementation}
    \subfile{report}
    \subfile{cd-pred-arg/cd-propositional}

    \bibliographystyle{plain}
    \bibliography{refs}
\end{document}