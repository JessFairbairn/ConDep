\documentclass{article}
\usepackage{booktabs}
\usepackage{graphicx}
\usepackage{subfiles}
\usepackage{natbib}
\usepackage{graphicx}
\usepackage{hyperref}
\usepackage{mathtools}
\usepackage{algpseudocode}
\usepackage{booktabs}
\usepackage{rotating}
\usepackage{pifont}
\usepackage{svg}
\usepackage{float}
\usepackage{listings}

\providecommand{\cdpred}{./cd-pred-arg} % relative path to master.tex
\providecommand{\repsfolder}{./reps}

\author{Jess Fairbairn}
\title{Dissertation}

\begin{document}
    \maketitle
    \tableofcontents
    \section{Introduction}
    This dissertation will begin by looking at conceptual dependency theory, a schema for expressing the semantics of natural language originally put forward by Roger Schank~\cite{SCHANK1972552}. This will then be compared to a number of other forms of information representation, with the aim of identifying what aspects of meaning each is able to capture.

    The code developed for this project will then be discussed, explaining how simulated events were ran in terms of conceptual dependencies. 

    \section{Background}
    \subfile{reps/paper.tex}
    % \subfile{conceptual_dependancies}

    \section{Aims}
    To plug the gap between some of the above knowledge representation systems, we aimed to build a system with the abilities of a numerical simulation, while maintaining the knowledge representation abilities of discrete events. To do this, Conceptual Dependencies were used as an intermediary between numerical data, natural language and visual representations- given how CD theory reduces actions to a small number of primitives, it simplifies the problem of simulations CD theory also has the advantage of having being developed with natural language in mind from day one, making this a natural stepping stone between the different knowledge representation regimes.

    It was also decided to explore the query answering capabilities of conceptual dependencies, using a logical programming module which could support querying about an input scenario.

    By bridging the gap between quantitative simulations and qualitative descriptions, we aim to ease both interpretation of results (especially the detection of certain events) and the setup of experiments. While this dissertation will discuss a simplistic implementation, it serves as a proof of concept for applying CD theory in a way which could interact with more realistic and powerful scientific models.

    Visual representation was processed by the Python library Pymunk, a two dimensional simulation library for rigid objects. Input events could be translated from natural language, into sequences of conceptual dependency events, then recreated in Pymunk. Alternatively, a Pymunk simulation can be launched without any pre-planned events, with changes in the objects expressed in terms of conceptual dependencies, which are then described in natural language. By describing physical events in terms of a small set of physical primitives we can make much more meaningful queries.


    \section{Implementation}
    \subfile{report}
    % \subfile{cd-pred-arg/cd-propositional}

    % \section{results}

    \section{Discussion}
    Defining verbs as CD scenarios seems to be broadly successful, with the primitive actions being sufficient to capture the meaning of words and to reconstruct events in a simulation scenario.

    The simulation to natural language pathway was successful, and could offer a technique for event recognition in future systems. An example would be using natural language to instruct a system what scenario to search for, setting the end condition for a program or experiment.

    Simulation systems could act as a research aid for scientists- by tapping into humans' ability to instinctively identify events in a visual scene, scientists could understand results quickly, identifying missing or unexpected features in a scenario.

    \section{Conclusion}
    \subsection{Future Work}
    % ML
    By breaking down verbs into a sequence of finite events, we open the possibility of reducing machine learning tasks regarding verbs to a question of learning and recognising patterns. Hierarchical Bayesian Program learning (BPL)~\cite{one-shot-learning} is built around combinations of primitives, and so offers a promising Machine learning regimes for a CD based system.
    
    This work required the creation of a library of verbs expressed as a sequence of CD events. These were manually input, a task which would be impractical for any system with a large dictionary. BPL offers `one shot' approaches to learning, promising a system to quickly learn from few examples.
    
    The recognition of primitives could be achieved in a variety of ways- here we have demonstrated a system which tracks entities in a simulation, thus having access to detailed mathematical information, but a visual recognition system to track objects and detect, say, an EMIT event seems perfectly feasible as well.

    % Necessary and sufficient
    This work took the initial CD primitives developed by Schank as a given, but there are other researchers who have questioned whether this set of primitives is necessary and sufficient to describe events~\cite{macbethimage}. We have already limited the set of primitives to physical ones, so the idea of altering the set isn't unreasonable. Situations which may imply the current set is inaccurate include:
    \begin{itemize}
        \item A physical scenario which can't be described by the current set of primitives (implying a new one is necessary)
        \item A physical scenario which can be expressed by multiple sets of primitives (potentially implying primitives can be combined, or there is overlapping meanings between the primitives which should be better demarcated)
    \end{itemize}

    Within the astrophysics domain, quantum and relativistic phenomena seem likely candidates for scenarios which can't be accurately represented by the present set of primitives.

    % More domains
    The use of conceptual dependencies shown in this dissertation could be used as a general purpose basis for describing physical actions, and is not restricted to the astrophysics domain used here. Different domains will pose different challenges though- for a micro-scale biological domain, chemical reactions would have to be taken into effect. These may be beyond the scope of the primitives described here, or may need to work on different levels of abstraction simultaneously.

    One potential field where action primitives may be useful is embodied robotics- if only the set of primitive actions need be hard-coded, complex verb definitions could be learnt in terms of sequences of primitives. Schank's original set of conceptual dependencies included GRASP for grabbing an object, as well as many abstract primitives for describing interactions with other people.

    % Extra work on aid for scientists
    While the work done here offers a promising proof of concept, it is limited in the kind of queries it can answer. One potentially useful approach would be for the system to automatically determine search spaces for a given question. Take the example question, `What is the maximum mass of a white dwarf before it collapses under gravity?'. It would have to infer that mass is the independent variable, and that there is an upper limit after which the described events occur. It would then need to run a series of simulations with different initial states (the mass of the white dwarf), possible using a bisecting algorithm or a similar appropriate mathematical approach. This kind of functionality would require greater question parsing abilities.

    A natural language parser which can construct Prolog queries would be a good asset for this project- at the moment, when using `Question Mode', the query must be written in Prolog, obviously a barrier for most users. Translating natural language questions into Prolog could end up being a complex task.

    % Object constraints
    The addition of object constraints for verbs could benefit both generation of natural language, and the parsing of input information. In his work on the BABEL natural language engine, \citeauthor{GOLDMAN1975289} used the example of `drink' and `eat'- two verbs which correspond to the same CD primitive (INGEST), but which are incorrect if used with the wrong object. This can be language dependent: Malay and Tagalog don't have gender distinctions for third person pronouns like English.

    As a more relevant example, the verbs `emit' and `eject' both correspond to the EXPEL primitive, but are used for different objects (you wouldn't say `it ejected radiation').

    These kinds of object constraints could also be useful for inferring information about unknown terms In the sentence `The star emits quaduool', we have the unfamiliar (nonsense, as it happens) term `quaduool'. But using the definition of the emit verb, we can infer it's some sort of radiation, and can animate it appropriately or extrapolate it's movements correctly.

    \bibliographystyle{plain}
    \bibliography{refs}
\end{document}