\documentclass[12pt,MSc,wordcount,twoside]{muthesis}
\usepackage{booktabs}
\usepackage{graphicx}
\usepackage{subfiles}
\usepackage{natbib}
\usepackage{graphicx}
\usepackage{hyperref}
\usepackage{mathtools}
\usepackage{algpseudocode}
\usepackage{booktabs}
\usepackage{rotating}
\usepackage{pifont}
\usepackage{svg}
\usepackage{float}
\usepackage{listings}

\providecommand{\cdpred}{./cd-pred-arg} % relative path to master.tex
\providecommand{\repsfolder}{./reps}


\begin{document}
    \title{Dissertation}
    \author{Jess Fairbairn}
    \principaladviser{Andr{\'e} Freitas}
        
    \beforeabstract

    \prefacesection{Abstract}
    An analysis of knowledge representation schemes is performed, systematically listing their properties and what forms of information they can capture and express. To try and bridge some of those information capturing and expression abilities, we develop a system which recognises or simulates events in terms of conceptual dependency theory, acting as a mediator between a numerical simulation and natural language descriptions of events.
    
    \afterabstract
    
    \prefacesection{Acknowledgements}
    I would like to my supervisor Andr{\'e} for guiding me through a project which has given me a chance to be creative and taught me about a wide range of computer science.

    Additionally, thank you to my parents for feeding me when I turn up at short notice, my friends Will, Ashleigh, Becka, Jenny and too many others to name who've helped me through a challenging year, Abbey and Angle the cats, Oliver the rabbit, and Our Lady Eris who never lets things get dull.
    \afterpreface
    

    \tableofcontents
    \section{Introduction}
    Knowledge representation is the area of computer science which looks at how best to capture and express information in a systematic way, so as to be used in automated applications.~\cite{brachman1992knowledge} It also ties into more fundamental questions about the nature of intelligence, whether that is artificial intelligence, or understanding how humans learn and think.

    This dissertation will begin by qualitatively comparing a number of forms of knowledge representation, with the aim of identifying what aspects of meaning each is able to capture. It will look in greater detail at conceptual dependency theory, a schema for expressing the semantics of natural language originally put forward by Roger Schank~\cite{SCHANK1972552}, as we explored it's use as a scientific aid and for representing concepts in code.

    The code developed for this project will then be discussed, explaining how simulated events were modelled in terms of conceptual dependencies. 

    \section{Background}
    \subfile{reps/paper.tex}
    % \subfile{conceptual_dependancies}


    \subsection{Comparing human concept learning to computer knowledge representation}
    There are interesting analogies between some of the schemas described above and theories about human cognition. We will now briefly discuss some theories about human concept learning, and relate them to computer knowledge representation methods. 

    \citealt{gardenfors2004conceptual} defined conceptual space as one or more domains expressing difference between concepts. Domains are a set of inseparable quantities which are needed together to understand a value. Researchers have tried to express the concept of `similarity' mathematically in terms of the conceptual distance between them- \citeauthor{shepard1987toward} estimated `similarity' to exponentually decay as concepts increased in difference: \(S_{ij}=e^{-Cd_{ij}}\).

    \subsubsection{Theories of concept learning}
    \begin{description}
        \item[Rule based concept learning] Rule based theories of concept learning assume humans develop knowledge about conceptual theories by constructing a series of rules to determine whether an instance belongs to a category or not. These will specify which criteria are necessary and/or sufficient to satisfy a categorisation.\cite{rouder2006comparing}
        \item[Prototype based concept learning] These theories assume humans memorise a `prototypical' instance of category, comparing other potential instances to that one.\cite{griffiths2007unifying} It measures `conceptual distance' from the prototype to judge if another example should be placed in the same category. An example may be a labrador being the prototypical dog, while other forms of dogs (such as chihuahuas) are still in that same category as long as they are close to the prototype.
        \item[Exemplar based concept learning] Exemplar based theories work on the basis that we store memories of examples we have seen from a category, and categorise new instances in terms of their proximity to that cluster of examples.\cite{griffiths2007unifying}
    \end{description}

    \citealt{rouder2006comparing} ran a psychological study to try and determine whether humans used rule or exemplar based logic for categorisation problems. Participants were asked to categorise squares by their size, for example putting small and large squares in category A, but medium sized squares in category B. Additionally, some of the participants had the task simplified by the inclusion of an on-screen ruler. Different patterns of  results were expected depending on which categorisation method they were using.
    
    The paper's conclusion was that the participants used a mixture of rule and exemplar approaches for the harder tasks, but once they were given a ruler, they exclusively used an exemplar approach. This indicated that humans use a rule based system only when the differentiation task becomes too difficult.

    In a similar combination of theories, more modern work has argued that prototype and exemplar theories are two extremes of a spectrum, with `multiple prototype' theories existing in between. To take an example- a Wooden Spoon for cooking and a standard metal spoon for eating are quite different- the distribution of the `spoon' area in difference space will not be isometric- take for example the property of bowl depth- a soup spoon has a deeper bowl relative to it's length, compared to other table top spoons. But if that spoon was proportionally expanded to the size of a wooden spoon used for cooking it would now fall out of the spoon zone into the ladle zone. To understand why both these kinds of spoon are part of the `spoon' category, there will have to be multiple prototypes- one for table spoons, and one for wooden spoons.

    \citealt{griffiths2007unifying} offers a mathematical treatment using Hierarchical bayesian approach- it models the distribution within a class as a number of clusters around \(K\) prototypes in difference space. The mathematical problem then becomes finding the optimum number of clusters and their positions to predict the spread of examples.

    \subsubsection{Relations to knowledge representation}
    There are analogies to be drawn between the concept learning theories discussed here and various knowledge representation schemes discussed above. Perhaps the most obvious is the connection between Prototype theories of concept learning and Frame based knowledge representation- indeed, some literature explicitly refers to `prototypes' in relation to frames, and prototypal inheritance (where an child instance of an object inherits all it's properties unless they are overwritten) is a term used to describe some object orientated languages like Javascript. These will work by spawning a new instance using the prototype's properties as default values, which can then be overwritten if needed on the instance.

    Rule based classification is obviously more analogous to first order logic languages- my our Prolog implementation of logic around conceptual dependencies (which is discuss in more detail below), we have a predicates such as `isEvent' and `isTime', which are inferred if a variable is used in certain ways.

    \section{Aims}
    To plug the gap between some of the above knowledge representation systems, we aimed to build a system with the abilities of a numerical simulation, while maintaining the knowledge representation abilities of discrete events. To do this, Conceptual Dependencies were used as an intermediary between numerical data, natural language and visual representations- given how CD theory reduces actions to a small number of primitives, it simplifies the problem of simulations CD theory also has the advantage of having being developed with natural language in mind from day one, making this a natural stepping stone between the different knowledge representation regimes.

    It was also decided to explore the query answering capabilities of conceptual dependencies, using a logical programming module which could support querying about an input scenario.

    By bridging the gap between quantitative simulations and qualitative descriptions, we aim to ease both interpretation of results (especially the detection of certain events) and the setup of experiments. While this dissertation will discuss a simplistic implementation, it serves as a proof of concept for applying CD theory in a way which could interact with more realistic and powerful scientific models.

    Visual representation was processed by the Python library Pymunk, a two dimensional simulation library for rigid objects. Input events could be translated from natural language, into sequences of conceptual dependency events, then recreated in Pymunk. Alternatively, a Pymunk simulation can be launched without any pre-planned events, with changes in the objects expressed in terms of conceptual dependencies, which are then described in natural language. By describing physical events in terms of a small set of physical primitives we can make much more meaningful queries.



    \section{Implementation}
    \subfile{report}
    % \subfile{cd-pred-arg/cd-propositional}

    % \section{results}

    \section{Discussion}
    Defining verbs as CD scenarios seems to be broadly successful, with the primitive actions being sufficient to capture the meaning of words and to reconstruct events in a simulation scenario.

    The simulation to natural language pathway was successful, and could offer a technique for event recognition in future systems. An example would be using natural language to instruct a system what scenario to search for, setting the end condition for a program or experiment.

    Simulation systems could act as a research aid for scientists- by tapping into humans' ability to instinctively identify events in a visual scene, scientists could understand results quickly, identifying missing or unexpected features in a scenario.

    \section{Conclusion}
    \subsection{Future Work}
    % ML
    By breaking down verbs into a sequence of finite events, we open the possibility of reducing machine learning tasks regarding verbs to a question of learning and recognising patterns. Hierarchical Bayesian Program learning (BPL)~\cite{one-shot-learning} is built around combinations of primitives, and so offers a promising Machine learning regimes for a CD based system.

    \citealt{one-shot-learning} uses the problem of handwriting recognition to demonstrate their technique- alphabetical symbols are split into character types (e.g.~specific letters) and character tokens (instances of a letter drawn by a single person). The types are defined in terms of their strokes (individual lines, separated by the pen leaving the page), sub-strokes (instances where the pen stops, but does not leave the page), and the relative spatial positions of strokes. In our compound verb definitions, the tokens would be analogous to a verb, strokes analogous to the primitive CD events which make up that verb (varying in number), and the relative positions of the strokes analogous to the order of the CD primitive events which are expressed in a verb.
    
    The work in this dissertation required the creation of a library of verbs expressed as a sequence of CD events. These were manually input, a task which would be impractical for any system with a large dictionary. BPL offers `one shot' approaches to learning, promising a system to quickly learn from few examples.
    
    The recognition of primitives could be achieved in a variety of ways- here we have demonstrated a system which tracks entities in a simulation, thus having access to detailed mathematical information, but a visual recognition system to track objects and detect, say, an EMIT event seems perfectly feasible as well.

    % Necessary and sufficient
    This work took the initial CD primitives developed by Schank as a given, but there are other researchers who have questioned whether this set of primitives is necessary and sufficient to describe events~\cite{macbethimage}. We have already limited the set of primitives to physical ones, so the idea of altering the set isn't unreasonable. Situations which may imply the current set is inaccurate include:
    \begin{itemize}
        \item A physical scenario which can't be described by the current set of primitives (implying a new one is necessary)
        \item A physical scenario which can be expressed by multiple sets of primitives (potentially implying primitives can be combined, or there is overlapping meanings between the primitives which should be better demarcated)
    \end{itemize}

    Within the astrophysics domain, quantum and relativistic phenomena seem likely candidates for scenarios which can't be accurately represented by the present set of primitives.

    % More domains
    The use of conceptual dependencies shown in this dissertation could be used as a general purpose basis for describing physical actions, and is not restricted to the astrophysics domain used here. Different domains will pose different challenges though- for a micro-scale biological domain, chemical reactions would have to be taken into effect. These may be beyond the scope of the primitives described here, or may need to work on different levels of abstraction simultaneously.

    One potential field where action primitives may be useful is embodied robotics- if only the set of primitive actions need be hard-coded, complex verb definitions could be learnt in terms of sequences of primitives. Schank's original set of conceptual dependencies included GRASP for grabbing an object, as well as many abstract primitives for describing interactions with other people.

    % Extra work on aid for scientists
    While the work done here offers a promising proof of concept, it is limited in the kind of queries it can answer. One potentially useful approach would be for the system to automatically determine search spaces for a given question. Take the example question, `What is the maximum mass of a white dwarf before it collapses under gravity?'. It would have to infer that mass is the independent variable, and that there is an upper limit after which the described events occur. It would then need to run a series of simulations with different initial states (the mass of the white dwarf), possible using a bisecting algorithm or a similar appropriate mathematical approach. This kind of functionality would require greater question parsing abilities.

    A natural language parser which can construct Prolog queries would be a good asset for this project- at the moment, when using `Question Mode', the query must be written in Prolog, obviously a barrier for most users. Translating natural language questions into Prolog could end up being a complex task.

    % Object constraints
    The addition of object constraints for verbs could benefit both generation of natural language, and the parsing of input information. In his work on the BABEL natural language engine, \citeauthor{GOLDMAN1975289} used the example of `drink' and `eat'- two verbs which correspond to the same CD primitive (INGEST), but which are incorrect if used with the wrong object. This can be language dependent: Malay and Tagalog don't have gender distinctions for third person pronouns like English.

    As a more relevant example, the verbs `emit' and `eject' both correspond to the EXPEL primitive, but are used for different objects (you wouldn't say `it ejected radiation').

    These kinds of object constraints could also be useful for inferring information about unknown terms In the sentence `The star emits quaduool', we have the unfamiliar (nonsense, as it happens) term `quaduool'. But using the definition of the emit verb, we can infer it's some sort of radiation, and can animate it appropriately or extrapolate it's movements correctly.

    \bibliographystyle{plain}
    \bibliography{refs}
\end{document}