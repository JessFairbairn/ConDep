\documentclass{article}
\usepackage{booktabs}
\usepackage{graphicx}
\usepackage{subfiles}
\usepackage{natbib}
\usepackage{graphicx}
\usepackage{hyperref}
\usepackage{mathtools}
\usepackage{algpseudocode}
\usepackage{booktabs}
\usepackage{rotating}
\usepackage{pifont}
\usepackage{svg}
\usepackage{float}

\providecommand{\cdpred}{./cd-pred-arg} % relative path to master.tex
\providecommand{\repsfolder}{./reps}

\author{Jess Fairbairn}
\title{Dissertation}

\begin{document}
    \maketitle
    \tableofcontents
    \section{Introduction}
    This dissertation will begin by looking at conceptual dependency theory, a schema for expressing the semantics of natural language originally put forward by Roger Schank~\cite{SCHANK1972552}. This will then be compared to a number of other forms of information representation, with the aim of identifying what aspects of meaning each is able to capture.

    The code developed for this project will then be discussed, explaining how simulated events were ran in terms of conceptual dependencies. 

    \section{Background}
    \subfile{reps/paper.tex}
    \subfile{conceptual_dependancies}

    \section{Implementation}
    \subfile{report}
    \subfile{cd-pred-arg/cd-propositional}

    % \section{results}

    \section{Discussion}
    Defining verbs as CD scenarios seems to be broadly successful, with the primitive actions being sufficient to capture the meaning of words and to reconstruct events in a simulation scenario.

    The simulation to natural language pathway was successful, and could offer a technique for event recognition in future systems. An example would be using natural language to instruct a system what scenario to search for, setting the end condition for a program or experiment.

    Simulation systems could act as a research aid for scientists- by tapping into humans' ability to instinctively identify events in a visual scene, scientists could understand results quickly, identifying missing or unexpected features in a scenario.

    \section{Conclusion}
    \subsection{Future Work}
    This work required the creation of a library of verbs expressed as a sequence of CD events.

    \bibliographystyle{plain}
    \bibliography{refs}
\end{document}